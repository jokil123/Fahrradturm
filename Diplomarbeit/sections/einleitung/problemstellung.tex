\subsection{Problemstellung}

\paragraph{Problem aus der Sicht von Fahrradfahrer:innen in Ballungsräumen:}
\bquote{Es besteht keine Möglichkeit ihr hochwertiges Fahrrad (Trend zu E-Bikes und hochpreisige Rennfahrrädern) sicher zu verwahren. Viele Fahrradbesitzer:innen weichen auf den eigenen Balkon oder sogar ihren Wohn- oder Schlafzimmer als Abstellplatz aus, weil das Risiko von Totalverlust durch Vandalismus oder Diebstahl im öffentlichen Raum zu hoch ist. Sobald sie ihr Fahrrad benutzen, d. h. damit in die Stadt oder zum nächsten Bahnhof (bzw. U-Bahn-Station) fahren, stehen sie vor demselben Problem.}{ltw_problemstellung}


\paragraph{Problem aus der Sicht von Bewohner:innen von Ballungsräumen:}
\bquote{Die Städteplanung nach dem 2. Weltkrieg stand unter dem Primat des PKWs als dem vorrangigen Verkehrsmittel. Dementsprechend wurden breite Straßen mit hohen Kapazitäten (und dementsprechendem Platzverbrauch) und vor allem auch der notwendige Parkraum geschaffen. Man kann davon ausgehen, dass ein PKW im Stadtverkehr max. 2 Stunden/Tag fährt, d. h. Straßenkapazität beansprucht, aber dementsprechend mindestens das 11-fache an Parkraumkapazität belegt. (2 Stunden Fahrzeit = 22 Stunden Standzeit pro Tag). Kurz- und mittelfristig wurde versucht, das Problem mit dem Bau von Tiefgaragen und Parkhäusern zu lösen, aber mit steigenden Grundstückspreisen wird dieser Lösung zunehmend Kostengrenzen (aus der Sicht der Anbieter) bzw. Parkticketpreisgrenzen (aus Sicht der PKW-Nutzer) gesetzt. Das Verhältnis zwischen Platzverbrauch eines PKWs (2,5 m x 4 m = 10 qm?) und beförderter Person (meistens nur 1 Person / PkW) ist schlicht und einfach unwirtschaftlich, d. h. für moderne Städte untragbar.}{ltw_problemstellung}
