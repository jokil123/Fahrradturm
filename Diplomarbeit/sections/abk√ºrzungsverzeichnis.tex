\section{Abkürzungsverzeichnis}

% definiert alle Abkürzungen
\begin{acronym}
  % \acro {Abkürzung}{Erklärung}{Langform}
  \acro{RBG}{Regalbediengerät}
  \acro{RFID}{Radio Frequency Identification}
  \acro{NFC}{Near Field Communication}
  \acro{API}{Programmierschnittstelle}{Application Programming Interface}
  \acro{REST}{Representational State Transfer}
  \acro{JSON}{JavaScript Object Notation}
  \acro{HTTPS}{Hypertext Transfer Protocol Secure}
  \acro{SBC}{Einplatinencomputer}{Single Board Computer. Ein Computer auf einer einzelnen Platine.}
  \acro{GPIO}{General Purpose Input Output}
  \acro{SPI}{Serial Peripheral Interface}
  \acro{I2C}{Inter-Integrated Circuit}{On-Board-Kommunikationsprotokoll für Peripheriegeräte. Gut für kurze Distanzen und niedrige Übertragungsraten geeignet}
  \acro{RAM}{Arbeitsspeicher}{Random Access Memory}
  \acro{Arm}{Acorn RISC Machine}{CPU Architektur}
  \acro{BAAS}{Backend as a Service}
  \acro{IoT}{Internet of Things}
  \acro{SDK}{Software-Entwicklungskit}{Software Development Kit}
  \acro{CAD}{Konstruktionsprogram}{Computer Aided Design}
  \acro{IDE}{Entwicklungsumgebung}{Integrated Development Environment}
  \acro{FDM}{Fused Deposition Modeling}
  \acro{SLA}{Stereolithographie}{Stereolithography}
  \acro{SLS}{Selektives Laser Sintern}{Selective Laser Sintering}
  \acro{IPC}{Interprozesskommunikation}{Interprocess Communication}
  \acro{Stdin}{Standardeingabe}{Standard Input. Die Standardeingabe ist die Eingabe, die ein Programm von der Tastatur oder einer Datei erhält.}
  \acro{Stdout}{Standardausgabe}{Standard Output. Die Standardausgabe ist die Ausgabe, die ein Programm auf dem Bildschirm oder in einer Datei ausgibt. Meist mit dem Monitor bzw. Terminal verbunden.}
  \acro{FIFO}{First In First Out}
  \acro{LIFO}{Last In First Out}
  \acro{GUI}{Grafische Benutzeroberfläche}{Graphical User Interface}
\end{acronym}