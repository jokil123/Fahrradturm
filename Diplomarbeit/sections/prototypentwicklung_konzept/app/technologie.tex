\subsubsection{Technologie}
Um die App zu entwickeln, wird ein \Gls{Framework} benötigt, das eine gemeinsame Codebasis bietet, um die App für iOS und Android zugänglich zu machen. Es soll eine zeitsparende und effiziente Entwicklung gewährt werden. Dafür bieten sich folgende Technologien an:\\

\paragraph{React Native}Dabei handelt es sich um eine \Gls{Open-Source}-Bibliothek, die von Facebook entwickelt wurde und auf JavaScript basiert. React Native ermöglicht die Entwicklung von mobilen Apps für iOS und Android mit einer Codebasis.(siehe \ref*{reactnative})
\paragraph{Xamarin} Xamarin ist eine plattformübergreifende Entwicklungsumgebung, die von Microsoft entwickelt wurde. Xamarin ermöglicht die Entwicklung von Apps für iOS, Android und Windows mit einer Codebasis und unterstützt die Programmiersprachen C\# und F\#. \cite{vergleich}
\paragraph{Flutter} Flutter ist eine von Google entwickelte \Gls{Open-Source}-Bibliothek, die auf der Programmiersprache Dart basiert. Flutter ermöglicht die Entwicklung von mobilen Apps für iOS, Android und das Web mit einer Codebasis.\cite{flutter}


\paragraph{Entscheidung für React Native}Für die Entwicklung der App verwendet das Projektteam das Framework React Native. Mit React Native benötigt man nur einen Code für die Entwicklung der App, die auf Android und iOS funktioniert. Es ermöglicht eine schnelle und unkomplizierte Entwicklung der App. Bei der Entscheidung für React Native, ist es außerdem bedeutend, dass das Projektteam bereits Erfahrungen mit React, dessen Funktionsweise ähnlich ist, gesammelt haben. React Native eignet sich zudem gut für die Integration von Firebase, welche sehr gut dokumentiert ist.\cite{reactnative}\\

\paragraph{Entscheidung für Expo}Das Projektteam hat sich bei der App für das Framework Expo (\ref{sec:expo}), welches auf React Native basiert, entschieden. Damit kann man die App problemlos auf iOS und Android testen und vorführen. Man muss dabei keine Entwicklungsumgebung einrichten und spart wiederum Zeit.\\ \\
Das Framework ist sehr gut dokumentiert und die verschiedenen Bibliotheken eignen sich für die benötigten Funktionen der App.\\
