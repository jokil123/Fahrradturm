\subsubsection{Technologie}
\paragraph{Vergleich React Native mit Xamarin}Für die Entwicklung der App kommen für das Projektteam die zwei Frameworks React Native und Xamarin in Frage. Da sich das Team uneinig war, wurden beide Technologien miteinander verglichen.\\

\begin{itemize}
    \item Xamarin nutzt als Basis die Programmiersprachen C\# und F\#, während React-Native auf Javascript setzt.
    \item React Native wird häufiger verwendet als Xamarin. Dadurch ist es besser dokumentiert und es gibt allgemein mehr Informationen dazu.
    \item Xamarin hat je nach Anwendungsfall eine bessere Performance als React Native.
    \item Beide bieten eine Integration von Firebase, welche gut dokumentiert ist.
\end{itemize}\cite{vergleich}

\paragraph{Entscheidung für React Native}Für die Entwicklung der App verwendet das Projektteam das Framework React Native. Mit React Native benötigt man nur einen Code für die Entwicklung der App, die auf Android und iOS funktioniert. Es ermöglicht eine schnelle und unkomplizierte Entwicklung der App. Bei der Entscheidung für React Native, war es außerdem bedeutend, dass das Projektteam bereits Erfahrungen mit React, dessen Funktionsweise ähnlich ist, gesammelt haben. React Native eignet sich zudem gut für die Integration von Firebase, welche sehr gut dokumentiert ist.\cite{reactnative}\\

\paragraph{Entscheidung für Expo}Das Projektteam hat sich bei der App für das Framework Expo, welches auf React Native basiert, entschieden. Damit kann man die App problemlos auf iOS und Android testen und vorführen. Man muss dabei keine Entwicklungsumgebung einrichten und spart wiederum Zeit.\\ \\
Das Framework ist sehr gut dokumentiert und die verschiedenen Bibliotheken eignen sich für die benötigten Funktionen der App.\\
