\subsubsection{Screens}
\begin{description}
    \item[Karte] Der Tab Karte dient zur Darstellung der verfügbaren Möglichkeiten zur Lagerung von Fahrrädern in der Umgebung. Außerdem soll die Verfügbarkeit von freien Plätzen dargestellt werden.
\end{description}
\begin{description}
    \item[Fahrradturm] Der Tab Fahrradturm dient zur Anzeige folgender Informationen:

          \begin{itemize}
              \item Karte mit Standort des Turmes
              \item Anzahl verfügbaren Lagerungsmöglichkeiten
              \item Bereits vom User bei diesem Turm gelagerte Gegenstände
              \item Möglichkeit zum Einlagern von einem weiteren Rad oder Gegenstand
          \end{itemize}
\end{description}

\begin{description}
    \item[Aktivität] Im oberen Teil vom Tab Aktivität werden die vom User gelagerten Gegenstände angezeigt. Auf der unteren Hälfte werden vergangene Buchungen mit weiteren Informationen aufgelistet.
\end{description}

\begin{description}
    \item[Einstellungen] Der Tab Einstellungen bietet die Möglichkeit zu verschieden Unterseiten zu gelangen.
\end{description}
\begin{description}
    \item[Services] Der Tab Services dient zum Darstellen verschiedener Dienste. Es gibt eine Möglichkeit zum Teilen eines Entsperrungscodes für ein gelagertes Fahrrad und die Eingabe eines solches Codes. Außerdem werden mehrere externe Dienstleistungsmöglichkeiten zum Beispiel für das Ausleihen eines E-Scooters oder zum Abgeben von Paketen.
\end{description}




