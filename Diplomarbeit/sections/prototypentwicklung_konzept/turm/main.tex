\subsection{Turm}

% Um die Auswirkungen der App besser darzustellen, wurde zusätzlich noch ein Modell des Turms in der Skala 1:15 gebaut. Dieses Modell wird anschließend mit LEDs bestückt, um den Status der jeweiligen Radbox darzustellen. Die Lichter werden von einem Raspberry PI 4b gesteuert welcher auf Änderungen in der Datenbank wartet und bei Anfraten mit dem App Client interagiert.

Um den Aus- und Einlagerungsvorgang besser darzustellen sollte ein Modell des Turms gebaut werden. Dieser sollte den Aktuellen Status (belegte und ausgelagerte Stellplätze) des Turmes darstellen und auf Anfragen des Clients reagieren. Es sollte sozusagen ein Prototyp des Turmes entstehen, welcher die Funktionalität des Turmes simuliert.

Abgesehen von der Hardware, sollte der Turm auch Software beinhalten um sozusagen als Backend für die App zu dienen. Jeder Turm sollte als eigenständige Einheit dienen und Anfragen selbstständig beantworten können.


\subfile{raspberry_pi.tex}

\subfile{technischer_aufbau.tex}

\subfile{3d_druck.tex}