\paragraph{V3 - Rust}
Die V3 Variante ist in Rust (\ref{sec:rust}) implementiert. Rust ist eine moderne Programmiersprache, die sich durch eine gute Performance und eine gute Developer Experience auszeichnet. Rust unterscheidet sich stark von Python, da es eine Systemsprache ist.

Gegenüberstellung Rust und Python:
\begin{table}[H]
  \centering
  \begin{tabular}{lcc}
    \textbf{Rust vs. Python}                      & \textbf{Rust}    & \textbf{Python}       \\
    \toprule
    \textbf{Typisierung}                          & Statisch         & Dynamisch             \\
    \textbf{Garbage Collection}                   & Manuell          & Automatisch           \\
    \makecell[l]{\textbf{Performance }                                                       \\(100mio Stellen Pi Benchmark \\\citev{programming_language_speeds})}
                                                  & 69.31s           & \makecell[l]{5851.53s \\(84x langsamer)} \\
    \textbf{\Gls{kompiliert}/\Gls{interpretiert}} & \Gls{kompiliert} & \Gls{interpretiert}   \\
    \textbf{Multi-Threaded}                       & Ja               & Nein                  \\
    \bottomrule
  \end{tabular}
  \caption{Gegenüberstellung Rust und Python}
  \label{tab:rust_vs_python}
\end{table}

Rust wurde für V3 gewählt, da das zuständige Projektmitglied mehr Erfahrung mit Rust hat als mit Python. Obwohl Rust komplizierter ist als Python, ist die Entwicklungsgeschwindigkeit durch die bessere Developer Experience deutlich höher.

\noindent V3 besitzt auch eine \ac{GUI}, die den aktuellen Zustand des Turms darstellt. Diese ist in \texttt{fltk\_rs} implementiert.

\noindent Architekturtechnisch ist V3 sehr ähnlich zu V2. Dennoch ist die Architektur immer noch zu kompliziert und somit die Entwicklungsgeschwindigkeit stark beeinträchtigt.
