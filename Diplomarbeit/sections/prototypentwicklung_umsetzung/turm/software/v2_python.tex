\paragraph{V2 - Python}
Die Python Version 2 wird komplett überarbeitet und viel einfacher und übersichtlicher gestaltet. Das Konzept der mehreren Auslagerungsplätze wird und nur ein Auslagerungsplatz ist implementiert.

\noindent V2 ist basierend auf einer \Gls{statemachine}. Die verschiedenen States sind in Abbildung \ref{fig:tower_controller_v1_state_machine} zu sehen.

\begin{figure}[H]
  \centering
  \includegraphics[width=0.9\textwidth]{images/tower_controller_v2_state_machine.png}
  \caption{State Machine des Tower Controller V2}
  \label{fig:tower_controller_v1_state_machine}
\end{figure}

\clearpage

\subparagraph{States:}
\begin{itemize}
  \item \textbf{Idle} - Der Turm ist im Idle State, wenn er nicht gerade eine Aktion ausführt. In diesem Zustand wird auf \Glspl{event} gewartet. Wenn ein \Gls{event} eintritt, wird der Zustand entsprechend angepasst.
  \item \textbf{Waiting for Bicycle Insertion} - Ein Nutzer hat den Einlagerungsprozess eines Fahrrads gestartet, der Turm wartet nun auf das Einlegen des Fahrrads in die Radbox und die Einlagerungsbestätigung (Drücken des Buttons) des Nutzers.
  \item \textbf{Storing Bicycle} - Das Fahrrad wurde in die Radbox eingelagert und der Nutzer hat die Einlagerungsbestätigung erteilt. Der Turm wartet nun auf das Schließen der Radbox und das Abschließen des Einlagerungsprozesses. Anschließend wird der Zustand wieder auf Idle gesetzt.
  \item \textbf{Retrieving Bicycle} - Ein Nutzer hat den Auslagerungsprozess eines Fahrrads gestartet, der Turm wartet nun auf das Abschließen des Auslagerungsprozesses und das Öffnen der Radbox.
  \item \textbf{Waiting for Bicycle Removal} - Der Turm wartet auf das Entfernen des Fahrrads aus der Radbox und die Auslagerungsbestätigung des Nutzers. Anschließend wird der Zustand wieder auf Idle gesetzt. Falls der Nutzer die Auslagerungsbestätigung innerhalb eines gewissen Zeitraums nicht erteilt, wird der Zustand auf Storing Bicycle gesetzt und das Fahrrad wird eingelagert.
  \item \textbf{Idle Animation} - Falls der Turm für eine gewisse Zeit im Idle Zustand ist, wird eine Animation ausgeführt. Diese Animation soll den Turm visuell ansprechender machen.
  \item \textbf{Exit} - Der Turm Controller wird beendet.
\end{itemize}

Änhlich wie V1 besitzt V2 auch eine \ac{GUI}, die den aktuellen Zustand des Turms darstellt. Diese wurde einfach von V1 übernommen und angepasst.

Die V2 Variante funktioniert gut und hätte womöglich auch architekturell funktioniert, jedoch gibt es Probleme mit Python selbst. Die Probleme stammen von zyklischen Importen und erlauben es nicht die Entwicklung weiterzuführen. Womöglich kann das Problem gelöst werden indem auf Type Hints verzichtet wird, jedoch beeinträchtigt dies die Developer Experience und Lesbarkeit des Codes stark. Aus diesem Grund wurde entschieden die Entwicklung von V2 einzustellen und auf eine andere Programmiersprache zu wechseln.