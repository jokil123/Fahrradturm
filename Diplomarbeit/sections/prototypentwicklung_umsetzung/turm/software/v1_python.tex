\paragraph{V1 - Python}
Die Pythonversion ist basierend auf einem Queue System, bei dem verschiedene Komponenten in verschiedenen \Glspl{thread} selbstständig Jobs und Events hinzufügen. Ein oder mehrere Job-Handler entnehmen sich anschließend Jobs aus der Queue und führen ausführen. Ein Job-Handler stellt jeweils eine Einlagerungsstation dar. Damit werdern mehrere verschiedene Auslagerungsstationen abgebildet.

\begin{figure}[H]
  \centering
  \includegraphics[width=0.5\textwidth]{images/tower_controller_v1.png}
  \caption{Datenströme des Tower Controller V1}
  \label{fig:tower_controller_v1}
\end{figure}

\ac{Stdin} und \ac{Stdout} werden verwendet, um mit dem Benutzer zu kommunizieren. \ac{Stdin} wird verwendet, damit ein Benutzer manuell Jobs zur Queue hinzufügen kann. Dies ist vorallem beim \Gls{debuggen} nützlich. \ac{Stdout} wird verwendet um dem Benutzer Informationen über den aktuellen Status des Turms zu geben. Eingaben sind eine blockierende Operation, deshalb wurde \texttt{aioconsole} verwendet um \ac{Stdin} asynchron abzurufen.

\bigskip

\noindent In einem weiteren \Gls{thread} kommuniziert ein \ac{GPIO} \Gls{listener} mit einem physikalischen Button, welcher zur Bestätigung des Einlagerungsvorgangs verwendet wird. Der \Gls{listener} wartet auf einen Tastendruck und fügt anschließend einen Job (\Gls{event}) zur \Gls{queue} hinzu.

\bigskip

\noindent Der wichtigste \Gls{listener} ist der Datenbanklistener, dieser wartet auf Änderungen in der Datenbank und fügt entsprechende Jobs zur Queue hinzu. Dieser \Gls{listener} ist essenziell für die Kommunikation zwischen der App und dem Turm. Der \Gls{listener} wird von der offiziellen Python Bibliothek für Firebase bereitgestellt.

\bigskip

\noindent Zusätzlich wird eine \ac{GUI} entwickelt, welche die aktuellen Zustände der einzelnen Radboxen darstellt. Diese \ac{GUI} wird ebenfalls von einem \Gls{thread} ausgeführt und aktualisiert sich automatisch, wenn sich der Zustand der Radboxen ändert. Die \ac{GUI} ist in \texttt{tkinter} implementiert und soll bei der Entwicklung helfen, falls das Programm nicht auf dem Raspberri Pi ausgeführt wird oder die Elektronik noch nicht fertig ist.

\bigskip

\noindent Die Python Version war zu Anfang vielversprechend, jedoch gibt es einige Probleme mit der Architektur, die zu einem komplizierten und unübersichtlichen Code führten. Die Architektur wird in der V2 komplett überarbeitet.
