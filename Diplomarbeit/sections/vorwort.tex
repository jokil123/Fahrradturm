\section*{\centering{Vorwort}}
In den letzten Jahren hat sich das Fahrradfahren zu einer immer beliebteren und umweltfreundlicheren Alternative zum Autofahren entwickelt. Besonders E-Bikes haben in den letzten Jahren an Beliebtheit gewonnen, da sie eine schnelle und effiziente Fortbewegungsmöglichkeit bieten. Jedoch sind Diebstahl und Vandalismus von teuren Fahrrädern ein großes Problem und es gibt zu wenig sichere und praktische Abstell- und Auflademöglichkeiten für Fahrräder. Hier kommt das automatische Fahrradparkhaus ins Spiel. Ein automatisches Fahrradparkhaus ist eine turmförmige Konstruktion, in der möglichst viele Fahrräder auf kleinstem Raum gelagert werden können. Die Vision sieht vor, die Fahrräder mittels automatischer Lagerroboter auf mehreren vertikalen Ebenen aufzuteilen, um die Platzeffizienz weiter zu steigern. Durch die Förderung dieser Technologie können E-Bikes attraktiver gemacht werden und dazu beitragen, die Nutzung des umweltfreundlichen Verkehrsmittels zu erhöhen.\\
Die Projektidee stammt von unserem Projektbetreuer Günter Hämmerle, der den Kontakt zu unserem Partnerunternehmen LTW Intralogistics herstellte. Das Thema interessierte uns sofort und wir entschieden uns rasch, die Aufgabenstellung im Zuge unserer Diplomarbeit auszuarbeiten.\\
Mit dem Thema „Konzeption eines intelligenten, automatischen Fahrradparksystems“ hatten wir sowohl eine herausfordernde Arbeit als auch ein breites Themengebiet. In den letzten Wochen und Monaten arbeiteten wir fleißig an der Durchführung und Dokumentierung der Arbeit und sind stolz, die Ergebnisse nun präsentieren zu können.\\