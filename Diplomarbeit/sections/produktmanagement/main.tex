\section{Projektmanagement}

Um größere Projekte erfolgreich zu meistern, ist es notwendig, Projektmanagement zu etablieren. Dieses Projektmanagement ist in der Lage, Ziele zu definieren, Projekte zu organisieren, Ressourcen zu managen, Qualität zu gewährleisten, Kommunikation zu etablieren und Risiken zu identifizieren. Um verschiedene Teile eines Projekts zu planen werden verschiedene Projektmanagementpläne verwendet. Das Projektteam hat sich entschieden, die auf den nächsten Seiten folgenden Projektmanagementpläne zu erstellen. \citev{projektmanagement}

\bigskip

\noindent Die Projektmanagementpläne wurden in Drawio \citev{drawio} und XMind 8 \citev{xmind8} erstellt.

\subfile{meilensteine.tex}

% \subfile{projektzieleplan.tex}

\clearpage
\subfile{projektauftrag.tex}

\clearpage
\subfile{projektorganigramm.tex}

\subfile{projektstrukturplan.tex}

\subfile{objektstrukturplan.tex}

\subfile{risikoanalyse.tex}

\clearpage
\subfile{projektumweltanalyse.tex}

\clearpage
\subfile{kanban.tex}

\clearpage
\subfile{projektfunktionendiagramm.tex}

\subfile{zeiterfassung.tex}

\subfile{gantt.tex}

% \subfile{projektabschlussbericht.tex}