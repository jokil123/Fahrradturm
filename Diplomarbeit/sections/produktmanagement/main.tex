\section{Projektmanagement}

Um größere Projekte erfolgreich zu meistern, ist es notwendig, Projektmanagement zu etablieren. Dieses Projektmanagement ist in der Lage, die Projektziele zu definieren, die Projektorganisation zu planen, die Projektressourcen zu managen, die Projektplanung zu erstellen und zu steuern, die Projektqualität zu gewährleisten, die Projektkommunikation zu organisieren, die Projektrisiken zu identifizieren und zu managen, die Projektänderungen zu steuern und die Projektverantwortung zu übernehmen. Um verschiedene Teile eines Projekts zu planen werden verschiedene Projektmanagementpläne verwendet. Das Projektteam hat sich entschieden die auf den nächsten Seiten folgenden Projektmanagementpläne zu erstellen. \cite{projektmanagement}

Die Projektmanagementpläne wurden in Drawio\cite{drawio} und XMind 8\cite{xmind8} erstellt.

\subfile{meilensteine.tex}

\subfile{projektzieleplan.tex}

\subfile{projektauftrag.tex}

\subfile{projekthandbuch.tex}

\subfile{projektorganigramm.tex}

\subfile{projektstrukturplan.tex}

\subfile{objektstrukturplan.tex}

\subfile{risikoanalyse.tex}

\subfile{projektumweltanalyse.tex}

\subfile{kanban.tex}

\subfile{projektfunktionendiagramm.tex}

\subfile{zeiterfassung.tex}

\subfile{gantt.tex}

\subfile{projektabschlussbericht.tex}