\subsection{Nutzwertanalyse}

Mithilfe einer Nutzwertanalyse wurden elf neue und vorhandene Systeme zur
Lagerung von Fahrrädern bewertet. Drei der bewerteten Systeme wurden von Grund
auf selbst konzipiert. Anschließend wurden die Systeme anhand 24 Kriterien mit 1-5
Punkten bewertet.

Schwer quantifizierbare Aspekte, wie z.B. die Nutzerfreundlichkeit oder
Ausfallsicherheit wurden basierend auf den Ergebnissen der Umfrage und durch
Abstimmungen innerhalb der Gruppe bewertet. Einfacher quantifizierbare Kategorien
wie Flächenverbrauch, Baukosten und die durchschnittliche Einlagerungszeit wurden
anhand des Logarithmus der Merkmalsausprägung mit 1-5 Punkten versehen.


\paragraph{Quantitative Kriterien} Die quantitativen Kriterien wurden aus verschiedenen Quellen (vorallem Herstellerseiten und Angebote) zusammengetragen und in Tabelle \ref{tab:quantitative_kriterien} dargestellt.

\paragraph{Qualitative Kriterien} Die qualitativen Kriterien wurden innerhalb des Teams festgelegt. Die qualitative Bewertung der einzelnen Kriterien kann in Tabelle \ref{tab:qualitative_kriterien} eingesehen werden.

\subfile{quantitative_kriterien_tabelle.tex}
\subfile{qualitative_kriterien_tabelle.tex}

\subfile{gegenüberstellung.tex}

Anhand der Ergebnisse dieser Nutzwertanalyse, wurden zwei vielversprechende
Modelle zur weiteren Bearbeitung gewählt. Diese Modelle werden in den nächsten Kapiteln näher betrachtet und beschrieben. Die Ausgewählten Modelle sind: \textbf{Rondell} und \textbf{Hochregallager}. Das Rondell ist eine Adaptierung bestehender Systeme, das Hochregallager ist ein neues System, welches auf herkömmlichen Hochregallagern aufbaut.