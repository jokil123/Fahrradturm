\subsubsection{Gegenüberstellung}

Anhand der ermittelten Parameter können die einzelnen Modelle miteinander verglichen werden. Zum Beispiel kann eine Gegenüberstellungsmatrix zwei oder mehr Parameter wie in der folgenden Abbildung \ref{fig:gegenueberstellung} darstellen.

Kategorien:
\begin{itemize}
  \item Investitionskosten / Fahrrad
  \item Platzverbrauch / Fahrrad
\end{itemize}

Modelle:
\begin{itemize}
  \item Rondell
  \item Paternoster
  \item Hochregallager
  \item Fahrradbox
  \item Fahrradständer
  \item Unterirdisches Fahrradparkhaus
\end{itemize}

Anhand dieser Kategorien und Modelle wurde ein Diagramm erstellt, in der die einzelnen nach den Kriterien bewerteten Modelle gegenübergestellt werden. Anhand dieser Matrix können Nutzbarkeit und Einsatzgebiete der einzelnen Modelle verglichen werden. Die Matrix erleichtert es entscheidungen über das geeignete System zu treffen und potenzielle Marklücken zu identifizieren.

\begin{figure}[h]
  \centering
  \includegraphics[width=0.8\textwidth]{images/gegenüberstellung_matrix.png}
  \caption{Gegenüberstellung der Modelle}
  \label{fig:gegenueberstellung}
\end{figure}