\subsubsection{Qualitative Kriterien}

Die qualitativen Kriterien wurden innerhalb des Teams festgelegt.

\pagestyle{empty}
\newgeometry{top=5mm, bottom=10mm}

\begin{landscape}
  \begin{longtable}{p{0.25\textwidth}rrrrrrrr}
    \toprule
    \large\textbf{Variante}                            &
    \ve{\textbf{Skalierbarkeit}}                       &
    \ve{\textbf{Diebstahlsicherheit}}                  &
    \ve{\textbf{Brandsicherheit}}                      &
    \ve{\textbf{Stärungssicherheit}}                   &
    \ve{\textbf{Nutzerfreundlichkeit}}                 &
    \ve{\textbf{Sichtbarkeit}}                         &
    \ve{\textbf{Gestaltbarkeit}}                       &
    \ve{\textbf{Zusatznutzen}}                                                         \\

    \midrule

    \textbf{Bestehende Systeme}                                                        \\
    Paternoster 20 Räder (Vlocker)                     & 2 & 5 & 2 & 2 & 4 & 4 & 4 & 3 \\
    Rondell (Wöhr Bikesafe)                            & 4 & 5 & 4 & 3 & 4 & 5 & 4 & 4 \\
    Rondell Unterirdisch (Eco Cycle)                   & 1 & 5 & 3 & 3 & 4 & 2 & 2 & 3 \\
    Fahrradbox                                         & 4 & 4 & 3 & 4 & 4 & 2 & 3 & 3 \\
    Fahrradständer                                     & 5 & 1 & 4 & 5 & 4 & 2 & 1 & 1 \\
    Unterirdischer Fahrradständer (Amsterdam Centraal) & 2 & 2 & 2 & 5 & 3 & 1 & 1 & 1 \\

    \midrule

    \textbf{Neue Systeme}                                                              \\
    Rondell Auslagerung von Außen                      & 4 & 5 & 4 & 4 & 4 & 5 & 3 & 4 \\
    Hochregallager                                     & 3 & 5 & 3 & 3 & 4 & 4 & 4 & 4 \\

    \bottomrule

    % This is a hack but idc
    \multicolumn{2}{c}{}                                                               \\

    \caption{Quantitative Kriterien}
    \label{tab:quantitative_kriterien}
  \end{longtable}
\end{landscape}

\pagestyle{plain}

\restoregeometry