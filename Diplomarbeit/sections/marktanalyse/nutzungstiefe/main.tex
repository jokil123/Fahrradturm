\subsection{Erweiterung der Nutzungstiefe}

Im Laufe der Arbeit wurde es offensichtlich, dass das alleinige Lagern von Fahrrädern das Potential des Systems zu stark limitieren würde. Dadurch kann eine größere Nutzungstiefe erreicht werden, da mehr Personen und Verwendungszwecke angesprochen werden.\\
Man könnte andere Unternehmen wie zum Beispiel Paketdienste oder Leihanbieter in das automatische Fahrradparkhaus einbinden.\\
Das Projektteam kann sich folgende Verwendungszwecke vorstellen, ohne dass das Fahrradparkhaus technische Änderungen benötigt: \\

\subsubsection{Vermieten von Fahrrädern}
Die Vermietung von Fahrrädern kann über das Fahrradparkhaus erfolgen. Die Vorteile sind Schutz vor äußeren Einflüssen, gute Sichtbarkeit und unabhängige Buchung des Fahrrades. \\

\subsubsection{Vermieten von E-Scootern}
Immer mehr Städte verbieten das Abstellen von E-Scooter auf dem Gehsteig.\cite{krutzler_wien_2022} Durch eine Kooperation mit einem Anbieter von E-Scootern, könnte das Ausleihen über das automatische Fahrradparkhaus erfolgen. Es bietet Vorteile für die Anbieterfirma, Nutzende der E-Scooter und der Allgemeinheit, da die E-Scooter niemanden auf dem Gehsteig behindern und sicher verwahrt sind.\\

\subsubsection{Abstellen von Gegenständen}
Ein weiterer Verwendungszweck ist die Abgabe von Gegenständen. Ein geeigneter Standort wäre zum Beispiel ein Bahnhof in einer touristischen Stadt. Dort könnten Touristen ihren Rucksack oder Koffer einlagern, wenn sie die Stadt erkunden möchten.\\

\subsubsection{Abgabe und Abholung von Paketen}
Das automatische Fahrradparkhaus könnte außerdem als Paketshop fungieren. Somit wird eine zeitunabhängige Abgabe bzw. Abholung von Paketen auf dem Arbeitsweg ermöglicht. Grundlage dafür ist eine Kooperation mit einem Paketdienst wie zum Beispiel DPD oder DHL.\\

\subsubsection{Abgabe zur Fahrradreparatur}
Ein weiterer Verwendungszweck ist, dass man ein Fahrrad zur Reparatur abgeben könnte. Dabei könnte man das Fahrrad am Morgen in Fahrradparkhaus lagern und am Abend wieder abholen. Für Benutzerinnen und Benutzer des Fahrradparkhauses, gibt es den Vorteil, dass sie keinen Umweg zur Fahrradwerkstatt machen müssen, und idealerweise das Fahrrad nicht vermisst wird, da bis zur Abholung die Reparatur schon erledigt ist. Die Fahrradwerkstatt profitiert durch neue Kunden und Kundinnen. Die Grundlage dafür ist, dass möglich ist, der Werkstatt den Zugang zum Fahrrad zu gewähren.\\