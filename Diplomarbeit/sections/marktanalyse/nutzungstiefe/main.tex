\subsection{Nutzungstiefe}

\begin{itemize}
  \item Am Anfang waren die Gepäckaufbewahrung und die Schließfächer....
        Kann höhere Nutzungstiefe erreicht werden, wenn man Fahrräder lediglich als ein Aufbewahrungsgut neben Gepäckstücken, Rollern, Scootern, Fahrradanhängern,Kinderwagen, nasser Regenkleidung usw. auffasst?
  \item Paketzustellung
        Könnte ein Lagerplatz am Knotenpunkt als temporäre Übergabebox für Paketzusteller und Empfänger vermietbar sein?
  \item Wäre es denkbar, dass einem die Paketdienstleister tagsüber seine Pakete einfach in die Box zum abgestellten Fahrrad stellen und man dann am Abend bei der Abholung seines Fahrrads die Pakete ohne Zusatzwege gleich mitnehmen kann?
  \item Kann man sonstige Dienstleister direkt an die Fahrradboxen anschließen (z.B. Fahrradservice, BIPA, SPAR, Fahrradkuriere,…)
  \item Kann ein Lagerplatz in der Art ausgestattet werden, dass er während der Lagerung Akkus lädt?
  \item Kann ein Lagerplatz zusätzlich als Übergabeplatz für angeschlossene Dienstleister genutzt werden (Gepäckstücke, Fahrradreparatur; Fahrradreinigung, Chemische Reinigung von Kleidung)?
  \item Können auf diesem Weg Leihräder zur Verfügung gestellt werden?
  \item Mit welchen Missbrauchsfällen muss bei so einem System gerechnet werden? Welche Vorkehrungen können dagegen getroffen werden?
\end{itemize}