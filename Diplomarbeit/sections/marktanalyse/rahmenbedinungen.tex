\subsection{Verkehrspolitische und kommerzielle Rahmenbedingungen}

Fahrradfahren hat zahlreiche Vorteile: Es ist wesentlich umweltfreundlicher und günstiger als die Benutzung eines PKWs und fördert die eigene Gesundheit. Die EU hat sich zum Ziel gesetzt, die Zahl der per Rad zurückgelegten Kilometer bis 2030 auf 312 Milliarden zu verdoppeln, um den Klimawandel zu bekämpfen und die Verkehrsbelastung in Städten zu reduzieren \citev{ludecke_eu_nodate}. Dabei spielen intelligente Fahrradparksysteme eine wichtige Rolle bei der Förderung des Fahrradverkehrs. Kommunen und Städte können die Schaffung von Fahrradparkhäusern unterstützen, indem sie Anreize für Unternehmen und Verkehrsverbünde bieten oder selbst in den Bau von Fahrradparkhäusern investieren.

\noindent Aktuelle Gesetze und Verordnungen in Österreich bevorzugen das Auto gegenüber dem Fahrrad. Besonders beim Bau von Wohnungen oder Betrieben ist die verordnete Mindestanzahl von Fahrradstellplätzen geringer als die Mindestanzahl von Stellplätzen für Kraftfahrzeuge \citev{leitfaden_vorarlberg}. So ist beim Bau von einem Supermarkt 1 Parkplatz pro 30m² Geschäftsfläche und nur ein Fahrradstellplatz pro 50m² erforderlich \citev{noauthor_ris_nodate}.
