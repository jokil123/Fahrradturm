\subsection{Verkehrspolitische und kommerzielle Rahmenbedingungen}

Fahrradfahren hat zahlreiche Vorteile: Es ist wesentlich umweltfreundlicher und günstiger als die Benutzung von einem Auto und fördert die eigene Gesundheit. Die EU möchte die Zahl der per Rad zurückgelegten Kilometer bis 2030 auf 312 Milliarden zu verdoppeln, um den Klimawandel zu bekämpfen und die Verkehrsbelastung in Städten zu reduzieren.\cite{ludecke_eu_nodate} Deshalb können Fahrradparkhäuser eine wichtige Rolle bei der Förderung des Fahrradverkehrs spielen. Kommunen und Städte können die Schaffung von Fahrradparkhäusern unterstützen, indem sie Anreize für Unternehmen und Verkehrsverbünde bieten oder selbst in den Bau von Fahrradparkhäusern investieren.\\
Aktuelle Gesetze und Verordnungen in Österreich bevorzugen das Auto gegenüber dem Fahrrad. Besonders beim Bau von Wohnungen oder Betrieben ist die verordnete Mindestanzahl von Fahrradstellplätzen geringer als die Mindestanzahl von Stellplätzen für Kraftfahrzeuge.\cite{leitfaden_vorarlberg} So ist beim Bau von einem Supermarkt 1 Parkplatz pro 30m² Geschäftsfläche und nur ein Fahrradstellplatz pro 50m² erforderlich.\cite*{noauthor_ris_nodate} \\ 
