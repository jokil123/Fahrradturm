\subsection{Kundensegmente}
Als Kundschaft des Systems kommen sowohl öffentliche als auch private Einrichtungen in Frage.

\subsubsection{Großunternehmen}
Häufig besteht ein beträchtlicher Abteil des CO²-Fußabrucks eines Unternehmens aus dem Arbeitsweg der Mitarbeitenden, die derzeit noch oft mit dem PKW zur Arbeit fahren. Viele Unternehmen bauen aufgrund der steigenden Anzahl der Mitarbeiter:innen, teure Parkplätze \citev{parkhere_versteckte_2020}. Ein durchschnittlicher Parkplatz benötigt wertvolle Flächen und der Bau eines Auto-Parkhauses kostet viel Geld. Ein automatisches Fahrradparkhaus bietet sicheres Verstauen der Fahrräder der Mitarbeiter:innen mit einem geringen Flächenverbrauch und Anwohner:innen profitieren von reduzierter Verkehrsbelastung.

\subsubsection{Wohnbau-Unternehmen}
Beim Bau von großen Wohnanlagen kann ein automatisches Fahrradparkhaus einen Fahrradraum ersetzen. Je größer das Projekt ist, desto rentabler ist der Bau eines Fahrradparkhauses. Dieses kann entweder freistehend oder an der Außenseite eines Wohngebäudes platziert werden.

\subsubsection{Öffentliche Einrichtungen}
Öffentliche Einrichtungen wie das Land Vorarlberg sind bereit, hohe Summen in den Ausbau der Radinfrastruktur zu investieren, um den motorisierten Verkehr zu reduzieren \citev{voralberginvestiert}. Ein automatisches Fahrradparkhaus macht die Verwendung von Fahrrädern attraktiver und kann auch für Einnahmen sorgen.

\subsection{Öffentliches Verkehrsnetz und -verbünde}
Ein automatisches Fahrradparkhaus kann bei einem größeren Bahnhof installiert werden. Es benötigt wenig Platz und bietet eine sichere, attraktive Aufbewahrung für Fahrräder. Damit werden mehr Menschen zur Nutzung von öffentlichen Verkehrsmitteln bewegt. Zusätzlich kann das System für weitere Zwecke, wie dem Ausleihen von Fahrrädern oder dem Verstauen von Koffern, verwendet werden.

