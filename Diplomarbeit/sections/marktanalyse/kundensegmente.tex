\subsection{Kundensegmente}
Als Kundschaft unseres Systems können wir uns sowohl öffentliche als auch private Einrichtungen vorstellen. \\

\subsubsection{Großunternehmen}
Häufig besteht ein beträchtlicher Teil des CO²-Fußabrucks eines Unternehmens durch die Mitarbeiter, die derzeit noch oft mit dem PKW zur Arbeit fahren. Viele Unternehmen bauen aufgrund der steigenden Anzahl der Mitarbeiter, teure Parkplätze. Ein durchschnittlicher Parkplatz benötigt wertvolle Flächen und der Bau eines Auto-Parkhauses kostet viel Geld. Ein automatisches Fahrradparkhaus bietet ein sicheres Verstauen der Fahrräder der Mitarbeiter mit einem geringen Flächenverbrauch und Anwohner profitieren durch weniger Verkehr.\\

\subsubsection{Wohnbau-Unternehmen}
Beim Bau von großen Wohnanlagen könnte ein automatisches Fahrradparkhaus einen Fahrradraum ersetzen. Je größer das Projekt ist, desto rentabler ist der Bau eines Fahrradparkhauses. Dieses könnte entweder freistehend oder an der Außenseite eines Wohngebäudes platziert werden.\\

\subsubsection{Öffentliche Einrichtungen}
Öffentliche Einrichtungen wie das Land Vorarlberg sind bereit, hohe Summen in den Ausbau der Radinfrastruktur zu investieren, um den motorisierten Verkehr zu reduzieren. Ein automatisches Fahrradparkhaus macht die Verwendung von Fahrrädern attraktiver und kann auch für Einnahmen sorgen.\\

\subsection{Öffentliches Verkehrsnetz und -verbünde}
Ein automatisches Fahrradparkhaus kann bei einem größeren Bahnhof verwendet werden. Es benötigt wenig Platz und bietet eine sichere attraktive Aufbewahrung für Fahrräder. Damit können mehr Menschen zur Nutzung von öffentlichen Verkehrsmitteln bewegt werden. Zusätzlich könnte das System für weitere Zwecke, wie dem Ausleihen von Fahrrädern oder dem Verstauen von Koffern, verwendet werden.\\


