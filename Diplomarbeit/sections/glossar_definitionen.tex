\usepackage{glossaries}
\usepackage[automake]{glossaries-extra}
\makeglossaries

% Beispiel für eine Definition
\newglossaryentry{latex}
{
  name=latex,
  description={Is a markup language specially suited for scientific documents}
}

\newglossaryentry{maths}
{
  name=mathematics,
  description={Mathematics is what mathematicians do}
}

\newglossaryentry{listener}
{
  name=listener,
  description={Ein Listener ist ein Programm, welches auf Ereignisse wartet und ahand dieser Ereignisse eine Aktion ausführt}
}

\newglossaryentry{thread}
{
  name=thread,
  description={Ein Thread ist ein Teil eines Programms, welches unabhängig von anderen Teilen des Programms ausgeführt wird}
}

\newglossaryentry{debuggen}
{
  name=debuggen,
  description={Debuggen ist das Finden von Fehlern in einem Programm wärend es ausgeführt wird}
}

\newglossaryentry{event}
{
  name=event,
  description={Ein Event ist ein Ereignis, welches in einem Programm auftritt}
}

\newglossaryentry{queue}
{
  name=queue,
  description={Eine Queue ist eine Datenstruktur, welche Daten in einer bestimmten Reihenfolge speichert. Eine queue fuktioniert nach einem \ac{FIFO} Prinzip}
}

\newglossaryentry{statemachine}
{
  name=state machine,
  description={Eine State Machine ist ein Programm, welches immer in einem von mehreren bestimmten Zuständen ist. Jeder Zustand kann auf ein Ereignis reagieren und einen anderen Zustand auslösen}
}

\newglossaryentry{sleep}{
  name=sleep,
  description={Sleep ist eine Funktion, welche das Programm für eine bestimmte oder unbestimmte Zeit anhält}
}