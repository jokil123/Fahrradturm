\usepackage{glossaries}
\usepackage[automake]{glossaries-extra}
\makeglossaries

% Beispiel für eine Definition
\newglossaryentry{latex}
{
  name=latex,
  description={Is a markup language specially suited for scientific documents}
}

\newglossaryentry{maths}
{
  name=mathematics,
  description={Mathematics is what mathematicians do}
}

\newglossaryentry{listener}
{
  name=listener,
  description={Ein Listener ist ein Programm, welches auf Ereignisse wartet und anhand dieser Ereignisse eine Aktion ausführt}
}

\newglossaryentry{thread}
{
  name=thread,
  description={Ein Thread ist ein Teil eines Programms, welches unabhängig von anderen Teilen des Programms ausgeführt wird}
}

\newglossaryentry{debuggen}
{
  name=debuggen,
  description={Debuggen ist das Finden von Fehlern in einem Programm wärend es ausgeführt wird}
}

\newglossaryentry{event}
{
  name=event,
  description={Ein Event ist ein Ereignis, welches in einem Programm auftritt}
}

\newglossaryentry{queue}
{
  name=queue,
  description={Eine Queue ist eine Datenstruktur, welche Daten in einer bestimmten Reihenfolge speichert. Eine queue fuktioniert nach einem \ac{FIFO} Prinzip}
}

\newglossaryentry{statemachine}
{
  name=state machine,
  description={Eine State Machine ist ein Programm, welches immer in einem von mehreren bestimmten Zuständen ist. Jeder Zustand kann auf ein Ereignis reagieren und einen anderen Zustand auslösen}
}

\newglossaryentry{sleep}{
  name=sleep,
  description={Sleep ist eine Funktion, welche das Programm für eine bestimmte oder unbestimmte Zeit anhält}
}

\newglossaryentry{wrapper}{
  name=wrapper,
  description={Ein Wrapper ist ein Programm, welches eine andere Software in eine andere Programmiersprache übersetzt}
}

\newglossaryentry{racecondition}{
  name=race condition,
  description={Eine Race Condition ist ein Fehler, welcher auftritt, wenn zwei oder mehrere Threads auf eine Ressource zugreifen und diese Ressource nicht synchronisiert ist}
}

\newglossaryentry{masterbranch}{
  name=master-branch,
  description={Der Master-Branch ist der Haupt-Zweig eines Git-Repositories, also der Zweig, welcher normalerweise von den Entwicklern verwendet wird}
}

\newglossaryentry{pullrequest}{
  name=pull-request,
  description={Ein Pull-Request ist eine Änderungsvorschlag, welcher von einem Entwickler an einen anderen Entwickler gestellt wird, um Änderungen an einem Git-Repository zu übernehmen}
}

\newglossaryentry{collection}{
  name=collection,
  description={Eine Collection bei Firestore ist eine Sammlung von verschiedenen Dokumenten. Eine Collection kann keine anderen Sammlungen oder direkte Daten enthalten}
}

\newglossaryentry{document}{
  name=document,
  description={Dokumente sind innerhalb einer Collection eindeutig, mit eigenen Namen oder durch die automatische Benennung von Firestore. Enthält die Daten einer Firestore Datenbank}
}


\newglossaryentry{Framework}{
  name=Framework,
  description={\enquote{\textit{Ein Framework ist eine Ansammlung von Werkzeugen, vorgefertigten Features in Form von Code-Schnipseln und Regeln, welche dabei helfen sollen, Softwareprojekte schneller umzusetzen.}} \citev{framework}}
}

\newglossaryentry{Open-Source}{
  name=Open-Source,
  description={Open-Source ist ein Software-Entwicklungsmodell, bei dem der Quellcode öffentlich ist und von jeder Person gratis weiterentwickelt und genutzt werden kann.\cite{opensource}}
}

\newglossaryentry{Mockup}{
  name=Mockup,
  description={Ein Mockup ist ein digital gestalteter Entwurf von einer App}
}

\newglossaryentry{mutex}{
  name=Mutex,
  description={Ein Mutex ist ein Objekt, das es mehreren Threads ermöglicht, auf eine gemeinsame Ressource zuzugreifen, ohne dass ein anderer Thread die Ressource während des Zugriffs verwendet.}
}

\newglossaryentry{Tab}{
  name=Tab,
  description={Ansicht in der App}
}

\newglossaryentry{Location-Pin}{
  name=Location-Pin,
  description={Ein Location-Pin ist ein Symbol, das den Standort eines Ortes oder Objekts auf einer Karte markiert.}
}

\newglossaryentry{Unix-Timestamp}{
  name=Unix-Timestamp,
  description={Ein Unix-Timestamp ist eine Art der Zeitdarstellung, die die Anzahl der Sekunden seit dem 1. Januar 1970 um 00:00:00 Uhr UTC angibt und häufig in Datenbanken verwendet wird.\citev{unix}}
}

\newglossaryentry{SDK}{
  name=SDK,
  description={SDK steht für Software Development Kit und ist eine Sammlung von Programmierwerkzeugen und Programmbibliotheken, die zur Entwicklung von Software dient \citev{sdk}}
}

\newglossaryentry{Array}{
  name=Array,
  description={Ein Array ist ein Datentyp, in dem beliebig viele Werte abgespeichert werden können. \citev{Array}}
}

\newglossaryentry{Switch-Selector}{
  name=Switch-Selector,
  description={Ein Switch-Selector ist ein interaktives Element in einer App, das es der anwendenden Person ermöglicht, zwischen verschiedenen Optionen oder Zuständen zu wechseln.}
}

\newglossaryentry{StackNavigator}{
  name=StackNavigator,
  description={Ein StackNavigator ist eine Art der Navigation in einer App. Dabei werden die verschieden Ansichten aufeinander \enquote{gestapelt und die anwendende Person kann vor und zurück navigieren.} \citev{stacknavigator}}
}

\newglossaryentry{TabNavigator}{
  name=StackNavigator,
  description={Ein TabNavigator ist eine Art der Navigation in einer App. Dadurch kann die anwendende Person zwischen verschiedenen Abschnitten in der App wechseln. \citev{tabnavigator}}
}

\newglossaryentry{BottomTabBar}{
  name=BottomTabBar,
  description={Die BottomTabBar ist eine Navigationsleiste, die sich am unteren Ende des Bildschirms befindet.}
}



