\subsection{\LaTeX{}}
\label{sec:latex}

\LaTeX{} ist ein Schriftsetzungs-System, ausgelegt für technische und wissenschaftliche Dokumente. \LaTeX{} ist eine Weiterentwicklung von \TeX{} und wurde 1984 zum ersten Mal veröffentlicht. Es ist der de facto Standard für wissenschaftliche Dokumente und wird, da es Open-Source ist, kontinuierlich weiterentwickelt. \citev{latex_project,latex_wiki}

\bigskip

\noindent Latex wurde vom Projektteam verwendet, um die schrifliche Arbeit zu erstellen. Es bietet viele \gls{quality-of-life} Features wie automatische Bild- und Tabellennummerierung, Referenzierung und Inhaltsverzeichnisse. Da ein \LaTeX{} Projekt in viele kleine Dateien aufgeteilt werden kann, ist es einfach ein Versionskontrollsystem wie Git zu verwenden. Dies erleichterte die Zusammenarbeit und die Verwaltung von Änderungen.