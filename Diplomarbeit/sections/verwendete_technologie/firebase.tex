\subsection{Firebase}

Firebase ist eine Plattform von Google, welche es Entwicklern ermöglicht, eine Vielzahl von Funktionen in ihre Applikationen zu integrieren. Diese Funktionen sind zum Beispiel Authentifizierung, Datenbanken, Cloud Messaging und Hosting. Die Plattform ist für kleine Projekte kostenlos und bietet eine kostenlose Testversion mit 1 GB Speicherplatz und 50.000 Anfragen pro Tag. Firebase vereinfacht das Entwickeln von mobilen und Webapplikationen, da die Entwickler sich nicht um die Infrastruktur kümmern müssen. Die Plattform ist in verschiedene Komponenten aufgeteilt, welche einzeln oder zusammen verwendet werden können. Firebase verwendet ein "pay as you go" Modell bei dem Nutzer pro Anfrage einen bestimmten Betrag bezahlen. Firebase ist teil von Google Cloud Platform.

\begin{figure}[h]
  \centering
  \includegraphics[width=0.5\textwidth]{images/firebase_logo}
  \caption{Firebase (CC Google \cite{cc})}
  \label{fig:firebase_logo}
\end{figure}

Um die Entwicklung weiter zu Vereinfachen bietet Google offizielle Firebase \ac{SDK} für viele gängigen Programmiersprachen an. Diese SDKs bieten eine einfache \ac{API}, welche die Kommunikation mit der Plattform vereinfacht. Die Api ist sehr detailiert dokumentiert, weshalb es auch SDKs für andere Programmiersprachen gibt.

Firebase vereinfacht zwar die urprüngliche Entwicklung, jedoch gibt es das problem von Vendor Lock-in. Dies bedeutet, dass die Entwickler auf die Plattform von Firebase angewiesen sind, da diese viel zu eng mit Applikationen integriert ist.
