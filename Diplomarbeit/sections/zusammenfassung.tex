\section{Zusammenfassung}

In einem ersten Gespräch beim Partnerunternehmen hat das Projektteam (Joshua Lung, Paul Hartmann und Lukas Madlener) schnell festgestellt, dass das Thema „automatisches Fahrradparkhaus“ interressant ist und hat die ersten Ziele gemeinsam formuliert.

\noindent Die Ziele sind die Durchführung einer Marktanalyse, Konzeptionierung eines Fahrradparkhauses, Entwicklung einer App zum Buchen von Abstellplätzen und das Erstellen eines Modells des Fahrradturms. Für die Ziele sind entsprechende Meilensteine und Termine definiert, damit die Projektarbeit strukturiert voran geht. Zusätzlich wurden die aufgewendete Arbeitszeit pro Person dokumentiert.

\bigskip


\noindent Ausgangspunkt der Arbeit ist die Marktanalyse. Dabei werden bereits bestehende Systeme zur Lagerung von Fahrrädern analysiert, Recherchen und Anfragen bei Unternehmen gemacht und Informationen über gesetzliche Richtlinien gesammelt. Außerdem wird eine Online-Umfrage durchgeführt, um die Anforderungen und Interessen von Radfahrer:innen zu erfassen.

\smallskip \noindent Anschließend werden zwei neue automatische Fahrradparksysteme konzeptioniert. Bei dem Hochregallager handelt es sich um ein neuartiges System, dass bisher noch nicht für die Lagerung von Fahrrädern verwendet wird. Das Rondell mit Auslagerung von außen ist eine verbesserte Version von bereits verwendeten Modellen.

\smallskip \noindent Als Hauptteil der praktischen Ausführung wird ein Prototyp des Turmes gebaut und eine App entwickelt. Mit der App kann man Fahrradparkhäuser in der Umgebung finden, einen Account erstellen und ein Fahrrad oder einen Gegenstand bei einem Fahrradparkhaus abgeben oder abholen. Dabei handelt es sich um ein Modell des konstruierten Fahrradparkhauses, welches mithilfe von LEDs die verfügbaren Plätze darstellt.

\smallskip \noindent Alle gesetzten Meilensteine wurden zeitgerecht erreicht. Da die Abgabe der Diplomarbeit bis zum 31.03 erfolgen muss, wurde die praktische Ausarbeitung des Prototyps am 20.02 beendet. Einzelne Funktionen wie die Möglichkeit, den Zugriff auf ein gelagertes Fahrrad zu gewähren, wurden aufgrund des strikten Zeitplanes nicht umgesetzt.

\bigskip


\noindent Mit der Arbeit ist ein Grundstein für den Bau eines Fahrradparkhauses gelegt. Die App kann mit gewissen Anpassungen später veröffentlicht und verwendet werden. Mit dem Prototypen kann man eine Buchung darstellen und vorzeigen.

\noindent Zusammengefasst ist das Projektteam sehr zufrieden mit dem Ergebnis der Arbeit. Die Arbeit besteht aus einem breiten Themengebiet und es wurden viele nützliche Ergebnisse erzielt und Erfahrungen gesammelt.
