\section{Zusammenfassung}

In einem ersten Gespräch beim Partnerunternehmen haben wir (Joshua Lung, Paul Hartmann und Lukas Madlener) schnell festgestellt, dass uns die Thematik interessiert, und haben die ersten Ziele gemeinsam fixiert. \\
Unsere Ziele waren die Durchführung einer Marktanalyse, die Konzeptionierung eines Fahrradparkhauses, die Entwicklung einer App zum Buchen von Abstellplätzen und das Erstellen eines 3D-Modells des Fahrradturms. Für die Ziele haben wir entsprechende Meilensteine und Termine definiert, damit wir die Projektarbeit strukturiert angehen können. Zusätzlich wurden die eingesetzten Zeiten für die einzelnen Projektabschnitte pro Person dokumentiert.\\

Ausgangspunkt unserer Arbeit war die Marktanalyse. Dabei wurden bereits bestehende Systeme zur Lagerung von Fahrrädern analysiert, Recherchen und Anfragen bei Unternehmen durchgeführt und Information über politische Richtlinien gesammelt. Außerdem wurde eine Online-Umfrage durchgeführt, um die Anforderungen und von Radfahrer/innen zu erfassen.\\
Anschließend wurden zwei neue automatische Fahrradparksysteme konzeptioniert. Rondell mit Auslagern von außen. Hochregallager.\\
Als Hauptteil der praktischen Ausführung wurde ein Prototyp des Turmes gebaut und eine App entwickelt. Mit der App kann man Fahrradparkhäuser in der Umgebung finden, einen Account erstellen und ein Fahrrad oder Gegenstand bei einem Fahrradparkhaus abgeben oder abholen. Dabei handelt es um ein Modell des konstruierten Fahrradparkhauses, welches mithilfe von LEDs die verfügbaren Plätze darstellt. \\
Alle gesetzten Meilensteine wurden zeitgerecht erreicht. Da die Abgabe der Diplomarbeit bis zum 31.03 erfolgen muss, wurde die praktische Ausarbeitung des Prototyps am 20.02 beendet. Einzelne Funktionen wie das Teilen von Schlüsseln, um den Zugriff auf ein gelagertes Fahrrad zu gewähren wurden aufgrund des strikten Zeitplanes nicht umgesetzt.\\

Mit der Arbeit wurde ein Grundstein für den Bau eines Fahrradparkhauses gelegt. Die App kann mit gewissen Anpassungen später veröffentlicht und verwendet werden. Mit den Prototypen kann man eine Buchung darstellen und vorzeigen.\\
Zusammengefasst sind wir sehr zufrieden mit dem Ergebnis unserer Arbeit. Die Arbeit besteht aus einem breiten Themengebiet und es wurden viele nützliche Ergebnisse erzielt und Erfahrungen gesammelt.
